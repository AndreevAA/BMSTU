\chapter{Экспериментальная часть}

В данном разделе будет проведено функциональное тестирование разработанного программного обеспечения. Так же будет произведено измерение временных характеристик каждого из реализованных алгоритмов. 

\section{Технические характеристики}

Технические характеристики устройства, на котором выполнялось исследование:

\begin{itemize}
	\item процессор: Intel Core™ i5-8250U \cite{i5} CPU @ 1.60GHz;
	\item память: 32 GiB;
	\item операционная система: Manjaro \cite{manjaro} Linux \cite{linux} 21.1.4 64-bit.
\end{itemize}

Исследование проводилось на ноутбуке, включенном в сеть электропитания. Во время тестирования ноутбук был нагружен только встроенными приложениями окружения рабочего стола, окружением рабочего стола, а также непосредственно системой тестирования.

\section{Временные характеристики}

В реализованном программном обеспечении особый интерес представляет количество сравнений, необходимое для нахождения записи в словаре по ключу, так как это характеристика прямо пропорциональна времени работы алгоритма. Поэтому, будет проводиться анализ количества сравнений каждого из ключей словаря в словаре, а так же -- время поиска ключа не находящегося в словаре. Для большей наглядности число записей в словаре возьмем равным $10000$. 

Поиск будет производиться каждым алгоритмом по отдельности, после чего будут представлены соответствующие гистограммы. Отметим, что, в общем случае, распределение носит достаточно "случайный"\ характер, так как исходный массив не обязательно является упорядоченным. В связи с этим, каждая из построенных гистограмм будет продублирована второй, отображающей данные по убыванию числа сравнений.

\clearpage

\subsection{Алгоритм полного перебора}

В данном алгоритме зависимости числа сравнений от количества элементов является линейной. На пример, для обнаружения первого элемента понадобиться 1 сравнение, для 2ого -- 2, а для $N$ -- $N$.  В связи с этим, поиск последнего элемента в массиве потребует $N = 10000$ сравнений. 

\imgw{brute}{ht!}{\textwidth}{Число сравнений для нахождения имени полным перебором}

На Рисунке \ref{img:brute} видна линейная зависимость числа сравнений от положения элемента в массиве.
\clearpage

\imgw{brute_sorted}{ht!}{\textwidth}{Число сравнений для нахождения имени полным перебором по убыванию}
\clearpage

\subsection{Алгоритм бинарного поиска}

В данном алгоритме сложность поиска равна $O(log(N))$, в связи с этим, на общей гистограмме будет наблюдаться достаточно произвольное распределение, зависящее от входного массива.

\imgw{bin}{ht!}{\textwidth}{Число сравнений для нахождения имени бинарным поиском}

На Рисунке \ref{img:bin} не наблюдается явной зависимости числа сравнений для поиска элемента от его значения. В связи с этим, стоит упорядочить элементы по числу сравнений и построить вторую гистограмму.

\clearpage

\imgw{bin_sorted}{ht!}{\textwidth}{Число сравнений для нахождения имени бинарным поиском по убыванию}

На Рисунке \ref{img:bin_sorted} видна зависимость числа сравнений от искомого элемента в данном массиве. Заметим, что в худшем случае число сравнений составило 24, а в лучшем - 1 (элемент находится ровно по середине массива). Таким образом, алгоритм бинарного поиска показывает в 416 раз более высокий результат в сравнении с алгоритмом полного перебора. Стоит отметить, что лучший случай алгоритма полного перебора и лучший случай алгоритма бинарного поиска отличаются, в связи с чем, сравнивать их не корректно.

\clearpage

\subsection{Алгоритм частотного анализа}

В данном алгоритме сложность поиска зависит не только от положения в отсортированном сегменте, но и от размера используемого словаря, так как поиск в нем осуществляется полным перебором. В данной лабораторной работе использовался английский алфавит, содержащий 26 букв. В связи с этим, худший случай потребует не менее 27 сравнений для обнаружения элемента в частотном анализе. Тем не менее, в среднем случае это не должно оказывать серьезного действия на число требуемых сравнений.

\imgw{freq}{ht!}{0.85\textwidth}{Число сравнений для нахождения имени частотным анализом}

Как и в случае с бинарным поиском, на Рисунке \ref{img:bin} не наблюдается явной зависимости числа сравнений для поиска элемента от его значения. В связи с этим, стоит так же упорядочить элементы по числу сравнений и построить вторую гистограмму.

\clearpage

\imgw{freq_sorted}{ht!}{0.85\textwidth}{Число сравнений для нахождения имени частотным анализом по убыванию}

Заметим, что на Рисунке \ref{img:bin_sorted} видно, что в худшем случае число сравнений составило 33, а в лучшем - 2 (элемент находится ровно по середине массива первого сегмента). Таким образом, алгоритм бинарного поиска показывает в 303 раз более высокий результат в сравнении с алгоритмом полного перебора. Стоит отметить, что лучший случай алгоритма полного перебора и лучший случай алгоритма частотного анализа поиска отличаются, в связи с чем, сравнивать их не корректно.

\section{Вывод}

Исходя из полученных данных, можно сделать вывод, что алгоритм поиска в словаре, использующий частотный анализ, является боле эффективным, чем алгоритм полного перебора лишь в ряде случаев, в остальных же, он является менее эффективным, в связи с использованием сегментации и бинарным поиском внутри сегмента.

Отдельно отметим, что алгоритм бинарного поиска требует, в целом, меньшего числа сравнений, в связи с чем является более эффективным, чем алгоритм с частотным анализом. Однако, алгоритм бинарного поиска требует сортировки всего входного массива, что, в среднем случаи имеет сложность $O(nlog(n))$, в связи с чем алгоритм бинарного поиска становится менее эффективным, чем алгоритм частотного анализа, сортирующий данные по сегментам.
