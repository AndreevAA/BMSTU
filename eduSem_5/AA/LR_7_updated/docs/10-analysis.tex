\chapter{Аналитическая часть}

В данном разделе представлены теоретические сведения о рассматриваемых алгоритмах.

\section{Алгоритм полного перебора}

Алгоритмом полного перебора называют метод решения задачи, при котором по очереди рассматриваются все возможные варианты исходного набора данных. В случае словарей будет произведен последовательный перебор элементов словаря до тех пор, пока не будет найден необходимый. сложность такого алгоритма зависит от количества всех возможных решений, а время работы может стремиться к экспоненциальному.

Пусть алгоритм нашел элемент на первом сравнении. Тогда, в лучшем случае, будет затрачено $k_0 + k_1$ операций, на втором -- $k_0 + 2k_1$, на $N$ -- $k_0 + Nk_1$. тогда, средняя трудоемкость может быть рассчитано по формуле \eqref{eq:brute}, где $\Omega$ - множество всех возможных случаев.

\begin{equation}
   \label{eq:brute}
   \sum_{i \in \Omega} p_i t_i = (k_0 + k_1) \frac{1}{N + 1} + (k_0 + 2k_1) * \frac{1}{N + 1} + \dots + (k_0 + Nk_1) * \frac{1}{N + 1} 
\end{equation}

Из \eqref{eq:brute}, сгруппировав слагаемые, получим итоговую формулу для расчета средней трудоемкости работы алгоритма:

\begin{equation}
   k_0 + k1(\frac{N}{N + 1} + \frac{N}{2}) = k_0 + k1(1 + \frac{N}{2} - \frac{1}{N + 1}) 
\end{equation}

\section{Алгоритм двоичного поиска}

Данный алгоритм применяется к заранее упорядоченным словарям. Процесс двоичного поиска можно описать при помощи шагов:
\begin{itemize}
    \item сравнить значение ключа, находящегося в середине рассматриваемого интервала (изначально -- весь словарь), с данным;
    \item в случае, если значение меньше (в контексте типа данных) данного, продолжить поиск в левой части интервала, в обратном - в правой;
    \item продолжать до тех пор, пока найденное значение не будет равно данному или длина интервала не станет равной нулю (означает отсутствие искомого ключа в словаре).
\end{itemize}

Использование данного алгоритма в для поиска в словаре в любом из случаев будет иметь трудоемкость равную $O(log_2(N))$ \cite{knut}. Несмотря на то, что в среднем и худшем случаях данный алгоритм работает быстрее алгоритма полного перебора, стоит отметить, что предварительная сортировка больших данных требует дополнительных затрат по времени и может оказать серьезное действие на время работы алгоритма. Тем не менее, при многократном поиске по одному и тому же словарю, применение алгоритм сортировки понадобится всего один раз.

\section{Алгоритм частотного анализа}

Алгоритм частотного анализа строит частотный анализ полученного словаря. Чтобы провести частотный анализ, нужно взять первый элемент каждого значения в словаре по ключу и подсчитать частотную характеристику, т.е. сколько раз этот элемент встречался в качестве первого. По полученным данным словарь разбивается на сегменты так, что все записи с одинаковым первым элементом оказываются в одном сегменте.

Сегменты упорядочиваются по значению частотной характеристики таким образом, чтобы к элементу с наибольшим значением характеристики был предоставлен самый быстрый доступ.

Затем каждый из сегментов упорядочивается по значению. Это необходимо для реализации бинарного поиска, который обеспечит эффективный поиск в сегмента при сложности $O(nlog(n))$

таким образом, сначала выбирается нужный сегмент, а затем в нем проводится бинарный поиск  нужного элемента. Средняя трудоемкость при длине алфавита $M$ может быть рассчитана по формуле \eqref{eq:freq}.

\begin{equation}
   \sum_{i \in [1, M]} (f_{select_i} + f_{search_i})
   \label{eq:freq}
\end{equation}

\section{Описание словаря}

В данной работе будет использован словарь вида $\{username: string, password: string\}$, что представляет собой базу данных о паролях пользователей. Поиск будет осуществляться по полю \texttt{username}.

\section{Вывод}

В данной работе стоит задача реализации поиска в словаре. были рассмотрены алгоритмы реализации данного поиска.

Входными данными для программного обеспечения являются:
\begin{itemize}
    \item словарь из записей, вида \[\{username: string, password: string\}\] для поиска по нему;
    \item ключ для поиска в словаре.
\end{itemize}

Выходными данными является найденная в словаре запись для каждого из реализуемых алгоритмов.
