\chapter{Технологическая часть}

\section{Выбор средств реализации}

В качестве языка программирования для реализации данной лабораторной работы был выбран язык Golang \cite{golang}. Данный выбор обусловлен тем, что я имею некоторый опыт разработки на нем, а так же наличием у языка встроенных высокоточных средств тестирования и анализа разработанного ПО.

\section{Требования к программному обеспечению}

\section{Сведения о модулях программы}

Данная программа разбита на модули:
\begin{itemize}
    \item \texttt{main.go} - файл, содержащий точку входа в программу;
    \item \texttt{types.go} - файл, содержащий определение пользовательских типов данных;
	\item \texttt{dict.go} - файл, содержащий реализации алгоритмов поиска в словаре.
\end{itemize}

На листингах \ref{lst:types} -- \ref{lst:dict5} представлен код программы.

\listingfile{types.go}{types}{Go}{Определение пользовательских типов данных}{linerange={2-14}}

\listingfile{main.go}{main}{Go}{Основной файл программы main}{linerange={12-58}}

\listingfile{dict.go}{dict}{Go}{Реализация алгоритмов поиска в словаре}{linerange={14-52}}
\clearpage

\listingfile{dict.go}{dict2}{Go}{Реализация алгоритмов поиска в словаре}{linerange={54-82}}

\listingfile{dict.go}{dict3}{Go}{Реализация алгоритмов поиска в словаре}{linerange={84-107}}

\listingfile{dict.go}{dict4}{Go}{Реализация алгоритмов поиска в словаре}{linerange={109-144}}
\clearpage

\listingfile{dict.go}{dict5}{Go}{Реализация алгоритмов поиска в словаре}{linerange={146-162}}

\section{Тестирование}

В рамках данной лабораторной работы будет проведено функциональное тестирование реализованного программного обеспечения.

Тестирование проводилось на словаре, содержащем следующие записи:
\begin{itemize}
    \item \texttt{\{username: "username\_01"; password:"password\_01"\}};
    \item \texttt{\{username: "username\_02"; password:"password\_02"\}}.
\end{itemize}

В Таблице \ref{tbl:tests} приведены соответствующие тесты.

\begin{table}[ht]
	\small
	\begin{center}
		\caption{Таблица тестов}
		\label{tbl:tests}
		\begin{tabular}{|c|l|l|}
			\hline
			\bfseries Ключ & \bfseries Результат & \bfseries Значение \\ \hline
			\texttt{username\_01} & (1) & \texttt{\{username: "username\_01"; password:"password\_01"\}} (1) \\\hline
			\texttt{username\_02} & (1) & \texttt{\{username: "username\_02"; password:"password\_02"\}} (2) \\\hline
			\texttt{mama\_mia} & \bfseries -- & \bfseries -- \\\hline
		\end{tabular}
	\end{center}
\end{table}

При проведении функционального тестирования, полученные результаты работы программы совпали с ожидаемыми. Таким образом, функциональное тестирование пройдено успешно.

\section{Вывод}

В данном разделе был реализованы вышеописанный алгоритмы. Было разработано программное обеспечение, удовлетворяющее предъявляемым требованиям. Так же были представлены соответствующие листинги \ref{lst:types} -- \ref{lst:dict5} с кодом программы. Кроме того, было проведено функциональное тестирование разработанного программного обеспечения.